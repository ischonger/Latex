\documentclass[a4paper,portrait,onecolumn,11pt, titlepage, tableofcontents]{article} % aufgabe 1,5,6
%%%%%%%%%%%%%%%%%%%%%%%% preamble %%%%%%%%%%%%%%%%%%%%%%%%

% aufgabe 10
\usepackage[utf8]{inputenc} % für deutsche Umlaute.
\usepackage[T1]{fontenc}
\usepackage[english,ngerman]{babel}

\usepackage{amsthm}
\newtheorem{aufgabe}{Aufgabe}
\usepackage{graphicx}

%\usepackage{showframe}

%aufgabe 2
\usepackage{geometry}
\geometry{top=3cm, left=3cm, right=3cm, bottom=3.5cm}

%aufgabe 3
\usepackage{fancyhdr}
\pagestyle{fancy}
\lhead{Ingo Schäfer}
\rhead{165220}
\cfoot{\thepage\ von \pageref{LastPage}}

% aufgabe 5
\title{Latex für Naturwissenschaftler}
\author{Ingo Schäfer 165220}
\date{\today}

% aufgabe 8
\usepackage{multicol}
\setlength{\columnseprule}{2pt}
\setlength{\columnsep}{1cm}

%aufgabe 9
\newcommand{\mycommand}[1]{\textbf{\ref{#1}}}

% aufgabe 12
\usepackage{ulem}

% aufgabe 13
\usepackage[onehalfspacing,singlespacing]{setspace}

% aufgabe 15
\usepackage{pifont}

% aufgabe 16
\usepackage{enumerate}

% aufgabe 18
\setlength{\skip\footins}{1cm}
\renewcommand{\footnoterule}{\vspace*{-3pt}\dots\dots\dots\dots\dots\dots\dots\dots\vspace*{2.6pt}}

% aufgabe 19
\newcommand{\marginlabel}[1]{\mbox{}\reversemarginpar\marginpar{\hspace{8mm}#1}}

% aufgabe 2
%\hyphenation{Gas\-chro\-ma\-to\-gra\-phie Mass\-en\-spek\-tro\-me\-trie}

% aufgaben für kapitel 3 (Abbildungen und Tabellen)
\usepackage{graphicx}
% aufgabe 23
\graphicspath{ {pics/} {pdf/} {src/} }

% aufgabe 25
\usepackage{afterpage}

\usepackage{placeins}
\usepackage{wrapfig}

\usepackage{float}
\restylefloat{figure}
\restylefloat{wrapfig}


\usepackage{subcaption}

\usepackage{booktabs}
\usepackage{multirow}

\usepackage{array}
\usepackage{tabularx}
\usepackage{tabulary}
\usepackage{longtable}

% aufgabe 32
\usepackage{dcolumn}
 \newcolumntype{d}[1]{D{.}{.}{#1}}
 \newcolumntype{e}[1]{D{.}{.}{#1}}
\usepackage{siunitx} 

\usepackage{colortbl}
\usepackage{xcolor}
\usepackage{color}
 \definecolor{LightGrey}{rgb}{0.95,0.95,0.95}
 \definecolor{LessLightGrey}{rgb}{0.85,0.85,0.85}
 \definecolor{mygreen}{rgb}{0,0.6,0}
 \definecolor{mygray}{rgb}{0.5,0.5,0.5}
 \definecolor{myvauve}{rgb}{0.58,0,0.82}
 \definecolor{myred}{rgb}{1,0,0}
 \definecolor{mygold}{rgb}{0.96,0.9,0}
 
\usepackage{rotating}

\usepackage{caption}


% für 4_Mathematik
\usepackage{amsmath}
\usepackage{amssymb}

% für 5_Informatik
\usepackage{algpseudocode} 
\usepackage{algorithmicx} 
\usepackage{algorithm}
\usepackage{listings}
% aufgabe 37
\usepackage{courier}

% für Naturwissen Publikationsformate
\usepackage{pdfpages}
%%%%%%%%%%%%%%%%%%%%%%%%   main   %%%%%%%%%%%%%%%%%%%%%%%%
\begin{document}

%aufgabe 46
\includepdf[pages=1-3]{Aufgabe46/g-brief/beispiel.pdf}

%aufgabe 5
\maketitle

%aufgabe 6
\tableofcontents
\thispagestyle{empty}
%\setcounter{page}{0}		NOTE: Übersicht wird mitgezählt, aber nicht nummeriert.

% aufgabe 7
\pagebreak
\section*{Stärken und Schwächen von \LaTeX\ } 
\addcontentsline{toc}{section}{Stärken und Schwächen von \LaTeX\ }
\begin{quote}
\noindent
\LaTeX \ funktioniert nicht nach dem What-you-see-is-what-you-get - Prinzip.
Das mag erstmal etwas klobig in der Handhabung wirken, da immer wenn man seinen Schrieb anschauen will, das .tex-file
kompilieren muss.
Auch ist \LaTeX \ nicht in jedem Fall zu empfehlen: Eben bei solchen Arbeiten, wo man ständig nach dem Layout schauen muss
(z.B. Pixelschubsen bei Präsentationen, welche auch mit \LaTeX \ kreiert werden können), kann das ständige Zwischenkompilieren
nervig werden.
Besonders frickelig ist das Arbeiten mit Tabellen und Ähnlichem.
\LaTeX \ ist eher dafür gedacht ein bestimmtes Layout immer wieder verwenden zu können.
Darin liegt auch seine Stärke.
Oder auch das Vieles von \LaTeX \ automatisch gemacht wird. Wenn ich zum Beispiel eine Überschrift ändere, muss ich nicht in dem 
.tex-file nach weiteren Vorkommnissen (bspw. das Inhaltsverzeichnis suchen).
\LaTeX \ übernimmt sofort die Änderungen.
Besonders stark ist \LaTeX \ um wissenschaftliche Arbeiten zu Papier zu bringen.
Das Arbeiten mit Quellen, Fußnoten und Zitaten ist in \LaTeX \ super einfach gehalten.
Mathematische Formeln lassen sich auch sehr bequem aufschreiben.
Da \LaTeX \ alle möglichen Formelzeichen auf Halde hat, kommt man nicht in die Verlegenheit nur ähnliche Zeichen verwenden zu müssen.
Außerdem ist das Formelschreiben in \LaTeX \ wesentlich weniger frickelig als bspw. in Word.
\end{quote}


\pagebreak
\section{Basics}					% aufgabe 6
\label{sec:basics}
\subsection{Syntax}								% aufgabe 6

\begin{aufgabe}
Richten Sie zun\"achst das \LaTeX\ Dokument ein. Nutzen Sie A4 Format,
hochkant, einspaltig, Schriftgr\"o\ss e 11\,pt.
\end{aufgabe}	


\begin{aufgabe}
Setzen Sie die Seitenr\"ander auf 3\,cm (oben, rechts, links) und 3,5\,cm
(unten).
\end{aufgabe}

\subsection{Erzeugen des Dokuments}						% aufgabe 6
\begin{aufgabe}
Passen Sie Kopf- und Fu\ss zeilen des Dokuments folgenderma\ss en an:
Kopfzeile links enth\"alt Ihren Namen, Kopfzeile rechts Ihre Matrikelnummer;
Fu\ss{}zeile mitte enth\"alt die Seitenzahl in der Form ``x von X''.
\end{aufgabe}


\begin{aufgabe}
F\"ugen Sie nach diesem Abschnitt einen Seitenumbruch ein.	
\end{aufgabe}

%aufgabe 4
\pagebreak % oder \clearpage


\subsection{Dokumentaufbau, Seitenlayout und Präambel}				% aufgabe 6
\begin{aufgabe}
F\"ugen Sie dem Dokument einen Titel f\"ur das Protokoll, Datum sowie Ihren
eigenen Namen und Matrikelnummer als Autor hinzu. Die Titelseite soll auf
einer einzelnen Seite erscheinen.
\end{aufgabe}	

\subsection{Dokumentsruktur}							% aufgabe 6
\begin{aufgabe}
Gliedern Sie das Protokoll in einzelne Abschnitte entsprechend des Ablaufs
des Seminars. F\"ugen Sie ein Inhaltsverzeichnis ein. Das Inhaltsverzeichnis
soll auf einer einzelnen Seite stehen.
\end{aufgabe}

\begin{aufgabe}
F\"ugen Sie einen \textbf{unnummerierte} Abschnitt (z.B.\ ``Was ist
\LaTeX?'') an den Anfang des Protokolls, in der Sie eine kurze (ca.\ 150
W\"orter) Zusammenfassung zu \LaTeX\ schreiben. Der Abschnitt soll im
Inhaltsverzeichnis auftauchen.
\end{aufgabe}

\begin{multicols}{2}
\begin{aufgabe}
Setzen Sie diese Aufgabe in ein zweispaltiges Format. Der horizontale Abstand zwischen den Spalten soll 1\,cm betragen. Die beiden Spalten sollen durch eine 2\,pt dicke vertikale Linie getrennt werden. 		
\end{aufgabe}
\end{multicols}


\subsection{Crossreferencing}							% aufgabe 6
\label{subs:Crossreferencing}
\begin{aufgabe}
\label{aufg:9}
F\"ullen Sie in folgendem Satz die mit \textbf{XX} gekennzeichneten Stellen
mittels Crossreferencing.
\end{aufgabe}
In Aufgabe~\mycommand{aufg:9} im Abschnitt~\mycommand{subs:Crossreferencing} auf Seite~\textbf{\pageref{subs:Crossreferencing}}
behandeln wir die Verwendung von Labels um unkompliziert innerhalb des
Dokuments zu referenzieren.


\pagebreak
\section{Textformatierung}				% aufgabe 6
\label{sec:textformatierung}
\subsection{Encoding}
\begin{aufgabe}
Verwenden Sie das richtige Input und Output Encoding und setzen Sie die Spracheinstellungen auf die neue deutsche Rechtschreibung.	
\end{aufgabe}

\begin{aufgabe}
\TeX en Sie folgenden Text so exakt wie m\"oglich nach.	
\end{aufgabe}
\noindent \underline{Ursprungstext:} \\
\noindent\includegraphics[width=\textwidth]{aufgabe11}

\noindent \underline{nachge\TeX t:}\\
\singlespacing
\indent Im Deutschen verwendet man \glqq Anführungszeichen\grqq \ um direkte Rede oder wörtliche Zitate zu kennzeichnen -- 
sie können aber auch verwendet werden, um Teile eines Textes hervorzuheben, zu denen man Stellung nehmen möchte.
Auch Akzente lassen sich mit \LaTeX\ schreiben.
So kann man z.B. ganz einfach das türkische Sprichwort \glqq Ö\u{g}renmenin ya\c{s}{\i} yoktur.\grqq \ schreiben  -- 
\glqq Zum Lernen ist keiner zu alt.\grqq

\subsection{Schriften}					% aufgabe 6
\begin{aufgabe}
\TeX en Sie folgenden Text so exakt wie m\"oglich nach.	
\end{aufgabe}

\noindent \underline{Ursprungstext:} \\
\noindent\includegraphics[width=\textwidth]{aufgabe12}

\noindent \underline{nachge\TeX t:}\\
\indent 
In \LaTeX\ kann man verschiedene Schriftarten, -größen und -formen verwenden.
Damit kann man z.B.\ wichtige Dinge \textbf{fett hervorheben} oder \underline{unterstreichen}, oder aber unwichtige Dinge
\sout{durchstreichen}.
Man kann außerdem \textsc{Eigennamen} und \texttt{Befehle} von normalem Text unterscheiden.
\\
\\
\noindent\hspace*{21mm}
\Huge{Und} \huge{man} \LARGE{kann} \Large{Text} \large{immer} \normalsize{kleiner} \small{werden} \footnotesize{lassen}
\dots

\pagebreak
\subsection{White Space}				% aufgabe 6
\normalsize
\begin{aufgabe}
\TeX en Sie folgenden Text so exakt wie m\"oglich nach.	
\end{aufgabe}

\noindent \underline{Ursprungstext:} \\
\noindent\includegraphics[width=\textwidth]{aufgabe13}

\noindent \underline{nachge\TeX t:}\\
\normalsize
%\begin{onehalfspacing}
\onehalfspacing
\indent Für \TeX \ ist ein Leerzeichen das gleiche wie ganz viele \hfill Leerzeichen.\\
\setlength{\parindent}{12mm} \indent 
Auch Tabs und einzelne Zeilenumbrüche\\
werden wie Leerzeichen behandelt. Leerzeichen nach Befehlen werden ignoriert.
Mehr als ein Zeilenumbruch markiert den Beginn eines neuen Absatzes.\\
\\
Leerzeichen zwischen Sätzen sind größer als zwischen Wörtern. Das Satzende wird durch\\
einen Punkt nach einem Kleinbuchstabe erkannt.\linebreak
Das wird zum Beispiel ein Problem, wenn der Satz mit einem Großbuchstaben endeT.\
Oder wenn man B.Sc.\ abkürzen will.
Den Umbruch zwischen Initialen, zum Beispiel F.~Schiller, muss man auch manuell verhindern.\\
\hspace*{4mm} Zwischen Zahl und Maßeinheit kommt in der Regel ein kleinerer Abstand.
Man kann auch Abstände erzwingen.
Dies ist z.B. ein 2cm breiter \hspace{2cm} Abstand.
%\end{onehalfspacing}

\pagebreak
\subsection{Listen}					% aufgabe 6
\begin{aufgabe}
\TeX en Sie folgenden Text so exakt wie m\"oglich nach.	
\end{aufgabe}

\noindent \underline{Ursprungstext:} \\
\includegraphics[width=\textwidth]{aufgabe14}

\noindent \underline{nachge\TeX t:}\\
\begin{singlespacing}
\hspace{-1.45cm}
\begin{tabular}{l l l}
Montag 		& 07.30 – 12.30 Uhr & \hspace{-0.9cm} 13.30 – 15.30 Uhr \\
Dienstag 	& 07.30 – 12.30 Uhr & \hspace{-0.9cm} 13.30 – 17.30 Uhr \\
Mittwoch 	& 07.30 – 12.30 Uhr & \hspace{-0.9cm} (Beratung) \\
Donnerstag 	& 07.30 – 12.30 Uhr & \hspace{-0.9cm} 13.30 – 17.30 Uhr \\
Freitag 	& \hspace{-0.74cm} ca. 07.30 – 12.30 Uhr &  \\
& und nach Vereinbarung & \\
\end{tabular}
\end{singlespacing}

\pagebreak
\begin{aufgabe}
Erstellen Sie folgende Liste.
\end{aufgabe}

\noindent \underline{Ursprungstext:} \\
\noindent\includegraphics[width=\textwidth]{aufgabe15}

\noindent \underline{nachge\TeX t:}\\
Diese Unterlagen werden bei der Beantragung des Personalausweises benötigt:

\begin{itemize}
 \item[\ding{51}] gültiges Indentitätsbild
 \item[\ding{51}] aktuelles Lichtbild:
  \begin{itemize}
   \item[>] aktuelle Aufnahme
   \item[>] Frontalaufnahme, kein Halbprofil-Bild
   \item[>] das Gesicht muss zentriert auf dem Foto erkennbar sein
   \item[>] die Augen müssen offen und deutlich sichtbar sein
   \item[>] der Mund muss geschlossen sein und der Gesichtsausdruck neutral
  \end{itemize}
\end{itemize}


\begin{aufgabe}
Generieren Sie eine 1\,cm einger\"uckte Liste, die mit (A1), (A2), (A3),\dots\ gelabelt ist. Schreiben Sie darunter einen Satz, der auf das erste Element der Liste verweist.
\end{aufgabe}

\begin{enumerate}[\hspace{1cm} ({A}1)]
 \item Hund \label{it:hund}
 \item Katze \label{it:katze}
 \item Maus
 \item Marder
 \item Dachs
 \item Bieber
\end{enumerate}

\noindent Der Hund (A\ref{it:hund}) ist ein Haustier, das Geruch erzeugt.

\pagebreak
\begin{aufgabe}\label{aufg:desc}
Erstellen Sie eine \textnormal{\texttt{description}} Liste, in der Sie kurz
(je ein Satz) \TeX, \LaTeX\ und Word (im Hinblick auf deren Unterschiede)
beschreiben.
\end{aufgabe}

\begin{description}
 \item[\TeX] TeX ist ein Textsatzsystem mit eingebauter Makrosprache.
 \item[\LaTeX] LaTeX ist ist ein Softwarepaket, das die Benutzung des Textsatzsystems \TeX mit Hilfe von Makros vereinfacht.
 \item[Word] Word von Microsoft ist ein WYSIWYG-Editor.
\end{description}

\subsection{Fußnoten}						% aufgabe 6
\begin{aufgabe}
\label{aufg:18}
\setcounter{footnote}{2}
\renewcommand{\thefootnote}{\fnsymbol{footnote}}
F\"ugen Sie dieser Aufgabe eine Fu\ss note hinzu. Die Fu\ss{}note soll mit dem dritten Fu\ss{}notensymbol
gekennzeichnet sein. Sie soll mindestens 1\,cm  vom Haupttext entfernt mittels einer gepunkteten Linie 
abgetrennt werden.\footnote{Dies ist die Fußnote für Aufgabe \ref{aufg:18}.}
\end{aufgabe}

Vielleicht sollte aber auch die dritte Nummerierung benutzt werden. 
Schließlich steht ja nirgendwo festgeschrieben, 
dass~\verb*|\ddagger| das dritte Symbol ist. 
\hspace{-0.1cm}\footnote[3]{Dann ist das hier eine andere Fußnote für Aufgabe \ref{aufg:18}.}

\subsection{Randnotizen}
\begin{aufgabe}
Nutzen Sie Randnotizen um diese Aufgabe links mit einem 3\,mm breiten und 8\,mm hohen Balken zu markieren.

\reversemarginpar
\marginpar{\rule[0pt]{3mm}{8mm}}
\end{aufgabe}

\subsection{Boxen und Silbentrennung}
\begin{aufgabe}
\TeX en Sie folgende 3\,cm breite eingerahmte Box so exakt wie m\"oglich nach und achten Sie dabei insbesondere auf die korrekte Silbentrennung.
\end{aufgabe}

\noindent \underline{Ursprung:} \\
\noindent\includegraphics[width=\textwidth]{aufgabe20}

\noindent \underline{nachge\TeX t:}\\
\indent \hspace{-0.7cm}
\fbox{\begin{minipage}{3cm}
  \begin{onehalfspacing}
   Gas\-chro\-ma\-to\-gra-\ \\phie\--Mass\-enspek-\ \\tro\-me\-trie
  \end{onehalfspacing}
\end{minipage}}

\subsection{Farben}						% aufgabe 6
\begin{aufgabe}
\"Andern Sie die Hintergrundfarbe nur dieser Seite in einen sehr hellen Pastellton (mindestens 90\,\% Wei\ss{}anteil). (Der Einfachheit halber k\"onnen Sie den n\"achsten Abschnitt auf einer neuen Seite beginnen.)
\end{aufgabe}
\pagecolor{LightGrey}

\pagebreak
\section{Abbildungen und Tabellen}			% aufgabe 6
\label{sec:floats}
\input{tex/3_Floats}

\pagebreak
\section{Mathematik}					% aufgabe 6
\label{sec:mathematik}
\singlespacing

\begin{aufgabe}\label{aufg:formulas}
\TeX en Sie den Text aus \texttt{aufgabe\ref{aufg:formulas}.pdf} so exakt wie m\"oglich nach.	
\end{aufgabe}

\setlength\parindent{18pt} Wenn A und B beliebige Ereignisse sind und $P(B) \leq 0$ ist, dann ist die bedingte Wahrscheinlichkeit von $A$,
vorausgesetzt $B$ (auch die Wahrscheinlichkeit von $A$ unter der Bedingung $B$, notiert als $P(A \mid B)$, definiert
durch:
\begin{displaymath}
 P(A \mid B) = \frac{P(A \cap B)}{P(B)}
\end{displaymath}
Darin ist $P(A \cap B)$ die Wahrscheinlichkeit, dass $A$ und $B$ gemeinsam auftreten. 
$P(A \cap B)$ wird gemeinsame Wahrscheinlichkeit, Verbundwahrscheinlichkeit oder Schnittwahrscheinlicheit genannt.

\newtheorem{mfze}{Theorem}

%\numberwithin{equation}{}
\begin{mfze}{\textbf{Multiplikationssatz für zwei Ereignisse:}}
\begin{equation}
 P(A \cap B) = P(A \mid B) \cdot P(B)
\end{equation}
\end{mfze}

Verallgemeinert man den obigen Ausdruck des Multiplikationssatzes, der für zwei Ereignisse gilt, erhält man den
allgemeinen Multiplikationssatz. Man betrachte dazu den Fall mit $n$ Zufallsereignissen $A_1$ , $A_2$ , $\ldots$ , $A_n$.

\vspace{0.5cm}
\noindent
\begin{equation*}
 \begin{array}{ccccccccc} 
  P\biggl( \bigcap\limits_{i=1}^{n} A_i \biggl) &
  = &
  P(A_1) &
  \cdot &
  \frac{P(A_1 \cap A_2)}{P(A_1)} &
  \cdot &
  \frac{P(A_1 \cap A_2 \cap A_3)}{P(A_1 \cap A_2)} &
  \cdots &
  \frac{P(A_1 \cap A_2 \cap \cdots \cap A_n)}{ P(A_1 \cap A_2) \cap \cdots \cap A_{n-1}) }\\
    &
   = &
   P(A_1) &
   \cdot &
   P(A_2 \mid A_1) &
   \cdot &
   P(A_3 \mid A_1) \cap A_2) &
   \cdots &
   P\Bigl( A_n | \cap_{i=1}^{n-1} A_i \Bigl) \\
 \end{array}
\end{equation*}

\setlength\parindent{18pt} Sind nur bedingte Wahrscheinlichkeiten und die Wahrscheinlichkeiten des bedingenden
Ereignisses bekannt, ergibt sich die totale Wahrscheinlichkeit von A aus

\begin{mfze}{\textbf{Gesetz der totalen Wahrscheinlichkeit:}}
\begin{equation}
 P(A) = P(A \mid B) \cdot P(B) + P(A \mid B^c) \cdot P(B^c),
\end{equation}
\end{mfze}
\noindent \textit{wobei $B^c$ das Gegenereignis zu $B$ bezeichnet.}\\
\\
\indent Wenn $A$ und $B$ stochastisch unabhängig sind, gilt:
\begin{equation*}
 P(A \cap B) = P(A) \cdot P(B),
\end{equation*}
was dann zu Folgendem führt:

\begin{mfze}{\textbf{Stochastische Unabhängigkeit:}}
\\
\textit{Egal, ob das Ereignis $B$ stattgefunden oder nicht stattgefunden hat, ist die Wahrscheinlich-
keit des Ereignisses $A$ stets dieselbe.}
\\
\begin{equation}
\begin{array}{cccclc}
  P(A \mid B) & = & \frac{P(A) \cdot P(B)}{P(B)}  & = & P(A) & \textit{bzw.} \\
  & & &  = &  P(A \mid B^c) & \\
\end{array}
\end{equation}
\end{mfze}

Für den Zusammenhang zwischen $P(A \mid B)$ und $P(B \mid A)$ ergibt sich direkt aus der
Definition und dem Multiplikationssatz:

\newtheorem{cor}[mfze]{Korollar}
\begin{cor}{\textbf{Der Satz von Bayes:}}
 \begin{equation}
  P(A \mid B) = \frac{P(B \mid A) \cdot P(A)}{P(B)}.
 \end{equation}

\noindent \textit{Dabei kann $P(B)$ im Nenner mit Hilfe des Gesetzes der totalen Wahrscheinlichkeit berechnet werden.}
\end{cor}


\begin{aufgabe}
\TeX en Sie folgende Matrix:
\end{aufgabe}

\noindent \underline{Ursprung:} \\
\hspace{1cm} \includegraphics[width=0.4\textwidth]{aufgabe35}

\noindent \underline{nachge\TeX t:}\\ \vspace*{10cm} 
\begin{math}
 \textit{A}_{m,n} =
 \begin{pmatrix}
  a_{1,1} & a_{1,2} & \cdots & a_{1,n}\\
  a_{2,1} & a_{2,2} & \cdots & a_{2,n}\\
  \vdots  & \vdots  & \ddots & \vdots \\
  a_{m,1} & a_{m,2} & \cdots & a_{m,n}\\
 \end{pmatrix}
\end{math}


\pagebreak
\section{Informatik}					% aufgabe 6
\label{sec:informatik}
\subsection{Pseudocode}							% aufgabe 6
\begin{aufgabe}
\TeX en Sie folgenden Pseudocode so exakt wie m\"oglich nach.\\

\noindent \underline{Ursprung:} \\
\noindent\includegraphics[width=\textwidth]{aufgabe36}	

\noindent \underline{nachge\TeX t:}\\
\begin{algorithm}
   \caption{Strict Consensus Merger}
    \begin{algorithmic}[1]
      \Function{scm}{tree $T_1$, tree $T_2$}
	\State $X \leftarrow \mathcal{L}(T_1) \cap \mathcal{L}(T_2)$
        \If {$|X| \geq 3$}	\Comment{\textit{Otherwise, the merged tree will be unresolved.}}
        \State calculate $T_{1|X}$ and $T_{2|X}$
        \State $T_X \leftarrow$ \Call{strictconsensus}{$T_{1|X},T_{2|X}$}
        \ForAll{removed subtrees of $T_1$ and $T_2$}
	  \If{collision}	\Comment{\textit{Subtrees of $T_1$ and $T_2$ attach to the same edge e in $T_X$}}
	    \State{Insert all colliding subtrees to the same point on $e$.}   \Comment{\textit{creates polytomy}}
	  \Else
	    \State{Reinsert subtree into $T_X$ without violating bipartitions in $T_1$ or $T_2$.}
	  \EndIf
	  \EndFor \\
	\ \  \  \ \ \ \ \ \ \Return{$T_X$}
	\EndIf
       \EndFunction
\end{algorithmic}
\end{algorithm}

\end{aufgabe}

\pagebreak
\subsection{Sourcecode}							% aufgabe 6
\begin{aufgabe}\label{aufg:sourceCode}
\TeX en Sie den Quelltext aus \texttt{aufgabe\ref{aufg:sourceCode}.pdf} so exakt wie m\"oglich nach.(Quelltext: \texttt{AbstractSCMAlgorithm.java})\\ 
\end{aufgabe}

\lstset{ 
  backgroundcolor=\color{white},
  basicstyle=\footnotesize\ttfamily,
  breakatwhitespace=false,
  breaklines=true,
  captionpos=b,
  commentstyle=\color{mygreen},
  deletekeywords={...},
  escapeinside={\%*}{*)},
  extendedchars=true,
  frame=single,
  keepspaces=true,
  keywordstyle=\color{myred},
  language=Java,
  morekeywords={*,...},
  numbers=left,
  numbersep=5pt,
  numberstyle=\tiny\color{mygray},
  rulecolor=\color{white},
  showspaces=false,
  showstringspaces=false,
  showtabs=false,
  stepnumber=1,
  stringstyle=\color{mygray},
  tabsize=4,
  title=\lstname,
  emph={@Override},
  emphstyle={\color{mygold}}
}

\lstinputlisting[language=java]{src/AbstractSCMAlgorithm.java}

\pagebreak
\section{Abschlussarbeiten}				% aufgabe 6
\label{sec:abschlussarbeiten}
\subsection{Große Projekte verwalten und Buchstruktur}

\begin{aufgabe}
Erstellen Sie das Grundger\"ust f\"ur eine eigene Abschlussarbeit. \TeX en
Sie daf\"ur das Layout der Beipsieldatei \texttt{book.pdf} so exakt wie
m\"oglich nach. Auf den Inhalt wird dabei keinen Wert gelegt (verwenden Sie
\texttt{Lorem Ipsum} Text), jedoch auf Titelseite, Kopf-/Fu\ss zeilen,
Seitenzahlen, Abbildung mit Verweis etc. Die Kapitel sollen als einzelne
Dateien vorliegen. Die weiteren Aufgaben dieser Section bearbeiten Sie bitte
an der erstellten Abschlussarbeit. 
\end{aufgabe}

\begin{aufgabe}
F\"ugen Sie der Abschlussarbeit im Anhang eine beliebige Seite aus Ihrem Protokoll hinzu. 
\end{aufgabe}

\begin{aufgabe}
Erm\"oglichen Sie es im Inhaltsverzeichnis per Link zu den jeweiligen Sections zu gelangen. Die Links sollen dabei einen dunklen Blauton haben.  
\end{aufgabe}

\subsection{Hyperlinks und Metadaten}
\begin{aufgabe}
F\"ugen Sie der Abschlussarbeit Metadaten zu Ihrer Person hinzu.
\end{aufgabe}

\begin{aufgabe}
Stellen Sie das Dokument so ein, dass beim \"Offnen das Inhaltsverzeichnis als Lesezeichen angezeigt wird. Die Seiten sollen Seite f\"ur Seite (kein kontinuierliches Scrollen) angezeigt werden. 	
\end{aufgabe}

\subsection{Literaturverzeichnis}
\begin{aufgabe}
Erstellen Sie eine Literaturdatenbank im \texttt{bib} Format mittels
\texttt{JabRef}. F\"ugen Sie mindestens ein Paper (z.B.\ \"uber PubMed), ein
Buch, sowie eine Ver\"offentlichung in einem Konferenzband (z.B.\ LNCS)
hinzu. Verwenden Sie Keys der Form \texttt{autor:YY}. Zitieren Sie alle
Dokumente Ihrer Literaturdatenbank in der Abschlussarbeit im Harvard Stil. Zitieren Sie das
Buch mit Verweis auf eine Seite. Zitate im Text sollen in runden Klammern
stehen. Zitate im Verzeichnis sollen folgendes Format haben:

\medskip
\noindent\includegraphics[width=\textwidth]{Literatur}
\end{aufgabe}


\pagebreak
\section{Naturwissenschaftliche Publikationsformate}	% aufgabe 6
\begin{aufgabe}
Erstellen Sie ein eigenes Paper auf Grundlage eines Templates\\
(\texttt{https://www.sharelatex.com/templates/journals}). Denken Sie sich
eine sinnvolle Gliederung aus und f\"ullen Sie inhaltlich alle Stellen die
m\"oglich sind. F\"ur den Rest verwenden Sie wieder \texttt{Lorem Ipsum}
Text. F\"ugen Sie dem Paper ein Literaturverzeichnis hinzu (die bereits
erstellte Literaturdatenbank kann hier wiederverwendet werden).
\end{aufgabe}

\begin{aufgabe}
Erstellen Sie einen Lebenslauf auf deutsch. W\"ahlen Sie ein Template, dass
Ihnen selbst gef\"allt
(\texttt{https://www.sharelatex.com/templates/cv-or-resume}). F\"ullen Sie die
Lebenslauf mit fiktiven Informationen oder f\"ugen Sie, wenn Sie m\"ochten,
ihre eigenen Informationen ein.

\end{aufgabe}

\begin{aufgabe}
Erstellen Sie einen Brief auf Grundlage der \texttt{g-brief} Klasse. Der
Brief soll als Deckblatt f\"ur das komplette Protokoll dienen und eine
Auflistung aller Anh\"ange enthalten. Relevante Informationen (Name,
Matrikelnummer, Mailadresse, Datum, Anschrift) sollen ausgef\"ullt sein;
alle weiteren Informationen k\"onnen ausgedacht oder weggelassen werden.
\end{aufgabe}

\pagebreak
\listoffigures				% aufgabe 27
\listoftables				% aufgabe 33

\label{LastPage}
\end{document}