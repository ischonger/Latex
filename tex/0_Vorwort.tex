\pagebreak
\section*{Stärken und Schwächen von \LaTeX\ } 
\addcontentsline{toc}{section}{Stärken und Schwächen von \LaTeX\ }
\begin{quote}
\noindent
\LaTeX \ funktioniert nicht nach dem What-you-see-is-what-you-get - Prinzip.
Das mag erstmal etwas klobig in der Handhabung wirken, da immer wenn man seinen Schrieb anschauen will, das .tex-file
kompilieren muss.
Auch ist \LaTeX \ nicht in jedem Fall zu empfehlen: Eben bei solchen Arbeiten, wo man ständig nach dem Layout schauen muss
(z.B. Pixelschubsen bei Präsentationen, welche auch mit \LaTeX \ kreiert werden können), kann das ständige Zwischenkompilieren
nervig werden.
Besonders frickelig ist das Arbeiten mit Tabellen und Ähnlichem.
\LaTeX \ ist eher dafür gedacht ein bestimmtes Layout immer wieder verwenden zu können.
Darin liegt auch seine Stärke.
Oder auch das Vieles von \LaTeX \ automatisch gemacht wird. Wenn ich zum Beispiel eine Überschrift ändere, muss ich nicht in dem 
.tex-file nach weiteren Vorkommnissen (bspw. das Inhaltsverzeichnis suchen).
\LaTeX \ übernimmt sofort die Änderungen.
Besonders stark ist \LaTeX \ um wissenschaftliche Arbeiten zu Papier zu bringen.
Das Arbeiten mit Quellen, Fußnoten und Zitaten ist in \LaTeX \ super einfach gehalten.
Mathematische Formeln lassen sich auch sehr bequem aufschreiben.
Da \LaTeX \ alle möglichen Formelzeichen auf Halde hat, kommt man nicht in die Verlegenheit nur ähnliche Zeichen verwenden zu müssen.
Außerdem ist das Formelschreiben in \LaTeX \ wesentlich weniger frickelig als bspw. in Word.
\end{quote}
